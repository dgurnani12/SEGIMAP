\documentclass[a4paper,12pt]{article}

\usepackage[utf8]{inputenc}

\usepackage[T1]{fontenc}

\usepackage{svg}
% Fix svg rendering
\DeclareUnicodeCharacter{00A0}{ }

\begin{document}

\begin{titlepage}
\begin{center}
    \bfseries
    \huge University of Ottawa
    \vskip.2in
    \large SEG 2105
    \vskip1in
    \Large Assignment 5
    \vskip1.5in
    \emph{\huge IMAP Client-Server Implementation Outline}
\end{center}

\vskip1.4in

\begin{minipage}{.25\textwidth}
    \begin{flushleft}
        \bfseries\large Professor:\par \emph{Prof. Timothy Lethbridge}
    \end{flushleft}
\end{minipage}
\hskip.4\textwidth
\begin{minipage}{.25\textwidth}
    \begin{flushleft}
        \bfseries\large Students:\par \emph{Nikita Pekin (7216924)} \\
        \par \emph{William Pearson (7204884)}
    \end{flushleft}
\end{minipage}
\end{titlepage}

\section*{Problem Statement}

The problem being solved is the storage, retrieval, and manipulation of electronic mail on clients and servers.

\section*{Requirements}

\section*{Use Cases}

We have one type of user. Herein referred to as a user. A user may use multiple mail clients; each mail client is a different piece of software which acts as a client to the mail server.

\begin{enumerate}

\item As a user, I want my messages to be secure against other users reading them.

  \begin{itemize}
  \item Requires LOGIN command
  \item STARTTLS would increase security.
  \end{itemize}

\item As a user, I want to receive and read my mail on both my laptop and my phone.

  \begin{itemize}
  \item Requires FETCH command
  \item The server must be able to handle multiple clients logged in as one user at the same time.
  \end{itemize}

\item As a user, I want to be able to delete messages. I want to be able to see which messages are new and which messages I have already read.

  \begin{itemize}
  \item Requires CLOSE command
  \item Requires the message to have various flags - Seen, Recent and Deleted
  \end{itemize}

\item As a user, I want to be able to create and delete mail folders. I want to be able to copy and move messages between folders.

  \begin{itemize}
  \item Requires COPY, CREATE and DELETE commands.
  \item Requires support for a hierarchy of mail folders.
  \end{itemize}

\item As a user, I want to be able to search my messages.

  \begin{itemize}
  \item Requires SEARCH command.
  \item This is probably a bonus.
  \end{itemize}

\end{enumerate}

\section*{Architecture}

The programming language used to implement the system will be Rust - the new programming language currently under active development by Mozilla.

Rust is a type-safe, memory-safe, concurrency-oriented systems language.
This means that Rust is a good choice for a native application like an IMAP client or server for several reasons.
Rust can ensure the client and server handle all possible IMAP states and command results, which ensures no crashes.
Rust is incredibly performant, rivalling C on various benchmarks, which means that the server and client will be incredibly fast.
Rust also provides a very powerful concurrency system, which will allow the server to handle very high numbers of connections simultaneously and easily, through the use of lightweight task threads.

Rust is a fairly young language and not yet at version 1.0, but the project is expected to reach 1.0 by 2015.
This means that until 2015, Rust is not stable and will have frequent breaking changes.
To avoid frequent code breakages, our code will be built against Rust v0.12 (the latest stable release).

We will use the Maildir format for persistent mail data storage.
This means that mail data will be persistent via the filesystem, and not via a database.

Our implementation will focus purely on the server aspect as IMAP (being a standardized protocol) allows the server to communicate with any pre-existing, compatible IMAP client.
Because only the server will be written, UI and message display will not be a factor in the design.

The server itself will be written entirely in Rust, and use the Rust compilation utilities: the Rust compiler (rustc), LLVM (the bytecode backend), the Cargo package manager (for dependency management).
Due to Rust and IMAP's platform agnosticism, the server binaries will be entirely platform agnostic - running in Windows, OS X, and Linux without issue.
The server will also be able to communicate with IMAP clients on any platform.

The user authentication information will be store in flatfiles on the disk in the TOML format.
This means that the rust-toml library will be used to parse the TOML files.
The authentication information will consist of the user's username (their email), and a salted hash of their password, encrypted with the bcrypt cryptographic hashing algorithm, through the rust-crypto library.

If time constraints permit, TLS authentication and encryption will be implemented using a suitable rust TLS library.
Until then, authentication and IMAP interfacing will occur in plain-text.
Typically, plain-text communication would be a glaring vulnerability in this scenario, but is considered acceptable as this software is not meant for public use.

\section*{UML Class Diagram}

\begin{figure}[H]
    \centering
    \includesvg[height=\textheight,width=\textwidth]{imapuml}
    \caption{UML Class Diagram}
\end{figure}

\section*{List of Messages}


\end{document}
